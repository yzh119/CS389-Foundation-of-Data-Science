%!TEX program = xelatex
%%%%%%%%%%%%%%%%%%%%%%%%%%%%%%%%%%%%%%%%%
% Short Sectioned Assignment
% LaTeX Template
% Version 1.0 (5/5/12)
%
% This template has been downloaded from:
% http://www.LaTeXTemplates.com
%
% Original author:
% Frits Wenneker (http://www.howtotex.com)
%
% License:
% CC BY-NC-SA 3.0 (http://creativecommons.org/licenses/by-nc-sa/3.0/)
%
%%%%%%%%%%%%%%%%%%%%%%%%%%%%%%%%%%%%%%%%%

%----------------------------------------------------------------------------------------
%	PACKAGES AND OTHER DOCUMENT CONFIGURATIONS
%----------------------------------------------------------------------------------------

\documentclass[paper=a4, fontsize=11pt]{scrartcl} % A4 paper and 11pt font size
\usepackage{xeCJK}
\usepackage{gensymb}
\usepackage[T1]{fontenc} % Use 8-bit encoding that has 256 glyphs
\usepackage{fourier} % Use the Adobe Utopia font for the document - comment this line to return to the LaTeX default
\usepackage[english]{babel} % English language/hyphenation
\usepackage{amsmath,amsfonts,amsthm} % Math packages

\usepackage{lipsum} % Used for inserting dummy 'Lorem ipsum' text into the template
\usepackage{enumerate}
\usepackage{sectsty} % Allows customizing section commands
\allsectionsfont{\normalfont\scshape} % Make all sections centered, the default font and small caps

\usepackage{fancyhdr} % Custom headers and footers
\pagestyle{fancyplain} % Makes all pages in the document conform to the custom headers and footers
\fancyhead{} % No page header - if you want one, create it in the same way as the footers below
\fancyfoot[L]{} % Empty left footer
\fancyfoot[C]{} % Empty center footer
\fancyfoot[R]{\thepage} % Page numbering for right footer
\renewcommand{\headrulewidth}{0pt} % Remove header underlines
\renewcommand{\footrulewidth}{0pt} % Remove footer underlines
\setlength{\headheight}{13.6pt} % Customize the height of the header
\numberwithin{equation}{section} % Number equations within sections (i.e. 1.1, 1.2, 2.1, 2.2 instead of 1, 2, 3, 4)
\numberwithin{figure}{section} % Number figures within sections (i.e. 1.1, 1.2, 2.1, 2.2 instead of 1, 2, 3, 4)
\numberwithin{table}{section} % Number tables within sections (i.e. 1.1, 1.2, 2.1, 2.2 instead of 1, 2, 3, 4)

\setlength\parindent{2em} % Removes all indentation from paragraphs - comment this line for an assignment with lots of text

%----------------------------------------------------------------------------------------
%	TITLE SECTION
%----------------------------------------------------------------------------------------

\newcommand{\horrule}[1]{\rule{\linewidth}{#1}} % Create horizontal rule command with 1 argument of height

\title{	
\normalfont \normalsize 
\textsc{Zhiyuan College, Shanghai Jiaotong University} \\ % Your university, school and/or department name(s)
\horrule{0.5pt} \\[0.4cm] % Thin top horizontal rule
\huge CS389: Foundations of Data Science Homework I\\ % The assignment title
\horrule{2pt} \\ % Thick bottom horizontal rule
}

\author{Zihao Ye} % Your name

\date{\normalsize\today} % Today's date or a custom date

\begin{document}

\maketitle % Print the title

\section*{Exercise 2.9}
\subsection*{I}
$$\textit{\#face}(k) = 2^{d-k}{d \choose k} $$
\subsection*{II}
$$\sum_{k=0}^{d}\textit{\#face}(k) = 3^d $$ 
\subsection*{III}
$$\textit{\#surface area} = \textit{\#face}(d - 1) = 2d $$
\subsection*{IV}
$$\textit{\#surface area} = 2^{d-1} \cdot 2d = 2^d\cdot d $$
\subsection*{V}
Let $\theta$ be the width of surface.
$$V(\textit{surface}) / V = 1 - (1-\theta)^d \geq 1 - e^{-\theta d}$$
When $d$ is large enough, no matter how small $\theta$ is, this ratio shall be close to $1$.
\section*{Exercise 2.10}
\subsection*{I}
$$\textit{incremental unit of area} = 4\pi \sin(2\theta) $$
\subsection*{II}
$$I(\theta) = \int_{0}^{\frac{\pi}{2}} 4\pi \sin(2\theta) d\theta = 4\pi \sin^2(x)$$
\subsection*{III}
$$I\left(36\degree\right) = \frac{5 - \sqrt{5}}{2}  $$

\section*{Exercise 2.11}
Since $V(d) = \frac{2}{d}\frac{\pi^{\frac{d}{2}}}{\Gamma\left(\frac{d}{2}\right)}$, to find its maximum, we consider the ratio:
$$\frac{V(d)}{V(d-1)} = \pi^\frac{1}{2}\left(1 - \frac{1}{d}\right)\frac{\Gamma\left(\frac{d-1}{2}\right)}{\Gamma\left(\frac{d}{2}\right)}  $$

$$\frac{\Gamma\left(\frac{d-1}{2}\right)}{\Gamma\left(\frac{d}{2}\right)} = \left\{\begin{array}{ll} 
\pi^{\frac{1}{2}}\left(1 - \frac{1}{2}\right)\left(1 - \frac{1}{4}\right)\cdots \left(1 - \frac{1}{d - 2}\right) & \textit{even}\\
2\pi^{-\frac{1}{2}} \left(1 - \frac{1}{3}\right)\left(1 - \frac{1}{5}\right)\cdots \left(1 - \frac{1}{d - 2}\right) & \textit{odd}
\end{array}\right.$$

The formulas above shows that no matter whether $d$ is even or odd, the radio decreases as $d$ grows, which implies we need to find the largest $d$ that the corresponding ratio is no less than $1$(from both cases: $d$ is even or odd).

After necessary computation, we derive that $d = 4, 5$ are the largest $d$'s that has a ratio $\frac{V(d)}{V(d-1)}\geq 1$ for even $d$ and odd $d$ respectively.

However, $V(5) > V(4)$, thus $5$-dimensional unit ball has largest volume, about $5.26$.
\section*{Exercise 2.12}
\subsection*{I}
$$V_2(d) = \frac{2}{d}\frac{\left(4\pi\right)^{\frac{d}{2}}}{\Gamma\left(\frac{d}{2}\right)}$$

As $d$ grows, the numerator's magnitude is far less than the dominator's magnitude, as a result, its volume would converge to $0$.
\subsection*{II}
$$V_c(d) = \frac{2}{d}\frac{\left(c^2\pi\right)^{\frac{d}{2}}}{\Gamma\left(\frac{d}{2}\right)}$$

It will converge to $0$ too, for the same reason as stated above.
\subsection*{III}

$$V_f(d) = \frac{2}{d}\frac{\left(f(d)^2\pi\right)^{\frac{d}{2}}}{\Gamma\left(\frac{d}{2}\right)}$$

By using {\it Stirling's Approximation}, we derive that
$$f(d) = \sqrt{\frac{d}{2\pi e}}\cdot \sqrt[d]{\pi d} $$
is a feasible solution for this task.
\section*{Exercise 2.13}
This is because the definition of ``volume'' changes as $d$ differs. A $d$-dimension ball with volume $V$, however, must have a width $1$ (like a cylinder) in $d+1$-dimension space to maintain its original volume $V$.
\section*{Exercise 2.14}
\begin{enumerate}
	\item $$\frac{A(4)}{2}$$
	\item $$V(3) $$
\end{enumerate}
\end{document}